% --- Template for thesis / report with tktltiki2 class ---
% 
% last updated 2013/02/15 for tkltiki2 v1.02

\documentclass[finnish]{tktltiki2}

% tktltiki2 automatically loads babel, so you can simply
% give the language parameter (e.g. finnish, swedish, english, british) as
% a parameter for the class: \documentclass[finnish]{tktltiki2}.
% The information on title and abstract is generated automatically depending on
% the language, see below if you need to change any of these manually.
% 
% Class options:
% - grading                 -- Print labels for grading information on the front page.
% - disablelastpagecounter  -- Disables the automatic generation of page number information
%                              in the abstract. See also \numberofpagesinformation{} command below.
%
% The class also respects the following options of article class:
%   10pt, 11pt, 12pt, final, draft, oneside, twoside,
%   openright, openany, onecolumn, twocolumn, leqno, fleqn
%
% The default font size is 11pt. The paper size used is A4, other sizes are not supported.
%
% rubber: module pdftex

% --- General packages ---

\usepackage[utf8]{inputenc}
\usepackage[T1]{fontenc}
\usepackage{lmodern}
\usepackage{microtype}
\usepackage{amsfonts,amsmath,amssymb,amsthm,booktabs,color,enumitem,graphicx}
\usepackage[pdftex,hidelinks]{hyperref}

% Automatically set the PDF metadata fields
\makeatletter
\AtBeginDocument{\hypersetup{pdftitle = {\@title}, pdfauthor = {\@author}}}
\makeatother

% --- Language-related settings ---
%
% these should be modified according to your language

% babelbib for non-english bibliography using bibtex
\usepackage[fixlanguage]{babelbib}
\selectbiblanguage{finnish}

% add bibliography to the table of contents
\usepackage[nottoc]{tocbibind}
% tocbibind renames the bibliography, use the following to change it back
\settocbibname{Lähteet}

% --- Theorem environment definitions ---

\newtheorem{lau}{Lause}
\newtheorem{lem}[lau]{Lemma}
\newtheorem{kor}[lau]{Korollaari}

\theoremstyle{definition}
\newtheorem{maar}[lau]{Määritelmä}
\newtheorem{ong}{Ongelma}
\newtheorem{alg}[lau]{Algoritmi}
\newtheorem{esim}[lau]{Esimerkki}

\theoremstyle{remark}
\newtheorem*{huom}{Huomautus}


% --- tktltiki2 options ---
%
% The following commands define the information used to generate title and
% abstract pages. The following entries should be always specified:

\title{Test Driven Developement-menetelmän tehokkuus ja laatu}
\author{Petri Pihlajaniemi}
\date{\today}
\level{Kandidaatintutkielma}
\abstract{Tiivistelmä.}

% The following can be used to specify keywords and classification of the paper:

%\keywords{avainsana 1, avainsana 2, avainsana 3}

% classification according to ACM Computing Classification System (http://www.acm.org/about/class/)
% This is probably mostly relevant for computer scientists
% uncomment the following; contents of \classification will be printed under the abstract with a title
% "ACM Computing Classification System (CCS):"
% \classification{}

% If the automatic page number counting is not working as desired in your case,
% uncomment the following to manually set the number of pages displayed in the abstract page:
%
% \numberofpagesinformation{16 sivua + 10 sivua liitteissä}
%
% If you are not a computer scientist, you will want to uncomment the following by hand and specify
% your department, faculty and subject by hand:
%
% \faculty{Matemaattis-luonnontieteellinen}
% \department{Tietojenkäsittelytieteen laitos}
% \subject{Tietojenkäsittelytiede}
%
% If you are not from the University of Helsinki, then you will most likely want to set these also:
%
% \university{Helsingin Yliopisto}
% \universitylong{HELSINGIN YLIOPISTO --- HELSINGFORS UNIVERSITET --- UNIVERSITY OF HELSINKI} % displayed on the top of the abstract page
% \city{Helsinki}
%


\begin{document}

% --- Front matter ---

%\frontmatter      % roman page numbering for front matter

\maketitle        % title page
\makeabstract     % abstract page

\tableofcontents  % table of contents

% --- Main matter ---

\mainmatter       % clear page, start arabic page numbering




\section{Johdanto}

%Yleistä, miksi testataan

Ohjelmistotestauksen tarkoituksena on parantaa ohjelman laatua havaitsemalla ja poistamalla virheitä ohjelmakoodissa. Testauksen avulla ei voida todistaa ohjelman olevan virheetön, mutta sillä voidaan todistaa virheiden olemassaolo. Ohjelmaa testataan erilaisilla syötteillä, jonka jälkeen ohjelman toimintaa verrataan odotettuun oikeaan lopputulokseen. Jos lopputuloksissa on eroa, on testi löytänyt virheen.   \cite{Muccini08}.

%Prosessista on paljon lisätietoa Whittakerin artikkelissa, ehkä sieltä lisää?

Testien kirjoittajan on tunnettava ohjelman rakenne ja toiminta pystyäkseen suunnittelemaan ja kirjoittamaan testejä. Testin kirjoittamiseni on vaikea ja aikaa vievä prosessi, ja se vaatii kehittäjältä hyviä taitoja. \cite{Whittaker00}

Ohjelmiston testaus suoritetaan eri vaiheissa riippuen valitusta ohjelmistokehitysmenetelmästä. Vuonna 1970 Winston W. Roycen määrittelemä \emph{vesiputousmalli} (Waterfall Model) kuvaa ohjelmistokehityksen viisivaiheisena prosessessina: ensin analysoidaan vaatimukset ja suunnitellaan koko ohjelma tarkasti, sitten kirjoitetaan varsinainen ohjelmakoodi ja lopuksi tuotettu ohjelmakoodi testataan. Vaiheittaisissa menetelmissä (Incremental Model) edellinen vesiputousmalli on jaettu useisiin pienempiin palasiin; koko ohjelmaa ei siis tarvitse suunnitella ja toteuttaa kerralla noudattaen vesiputousmallin järjestystä, vaan ohjelman rakentuu pienemmistä osista jotka on suunniteltu ja toteutettu erikseen. \emph{Stoica et al} mukaan testaus on helppoa inkrementaalisissa menetelmissäi, kun taas vesiputousmallissa testauksessa havaittujen virheiden korjaaminen voi olla hankaa. Ketterät (Agile Model) ohjelmistokehityksen menetelmät perustuvat inkrementaaliseen malliin. \cite{Stoica13}. Yksi erityisesti testaamiseen keskittyvät toimintapa on ketteriin malleihin perustuva \emph{Test Driven Development}. \cite{Crispin06}

Aion tutkielmassani tarkastella TDD:n toimivuutta kahdella mittarilla: laadulla ja tehokkuudella. Nämä mittari ovat erittäin epämääräisiä, joten ensin on tutkittava mitä ne oikeastaan tarkoittavat.


\section{Ohjelmistotestauksen mittareita}

\subsection{Laatu}

\subsection{Tehokkuus}

\section{Test Driven Development}

%Mikä se on, kuka keksi, miksi keksi? Paljonko käytetään?

\subsection{Mitä se on? Miten eroaa muista?}


\subsection{Tutkimuksia}

%Tähän erilaisia tutkimuksia, mitä mittareita niissä on käytetty, millaiset tutkimusryhmät, mitä tuloksia



\subsection{Tuloksia}
%Toimiiko vai ei?
\subsubsection{Laatu}
\subsubsection{Tehokkuus}


\subsection{Meta-analyysit}

%Millaisia näkemyksiä meta-analyyseilla aiheesta

\section{Käyttö yritysmaailmassa}

%Käyttääkö tätä joku, ehkä jotain yritysten papruja aiheesta







% --- References ---
%
% bibtex is used to generate the bibliography. The babplain style
% will generate numeric references (e.g. [1]) appropriate for theoretical
% computer science. If you need alphanumeric references (e.g [Tur90]), use
%
% \bibliographystyle{babalpha-lf}
%
% instead.

\bibliographystyle{babalpha-lf}
\bibliography{../ref}




% --- Appendices ---

% uncomment the following

% \newpage
% \appendix
% 
% \section{Esimerkkiliite}

\end{document}